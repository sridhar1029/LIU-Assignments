\documentclass[a4paper]{article}
\usepackage{lmodern}
\usepackage{amssymb,amsmath}
\usepackage{ifxetex,ifluatex}
\usepackage{fixltx2e} % provides \textsubscript
\ifnum 0\ifxetex 1\fi\ifluatex 1\fi=0 % if pdftex
  \usepackage[T1]{fontenc}
  \usepackage[utf8]{inputenc}
\else % if luatex or xelatex
  \ifxetex
    \usepackage{mathspec}
  \else
    \usepackage{fontspec}
  \fi
  \defaultfontfeatures{Ligatures=TeX,Scale=MatchLowercase}
\fi
% use upquote if available, for straight quotes in verbatim environments
\IfFileExists{upquote.sty}{\usepackage{upquote}}{}
% use microtype if available
\IfFileExists{microtype.sty}{%
\usepackage{microtype}
\UseMicrotypeSet[protrusion]{basicmath} % disable protrusion for tt fonts
}{}
\usepackage[margin=1in]{geometry}
\usepackage{hyperref}
\hypersetup{unicode=true,
            pdftitle={Minimizing Programming Assignment Plagiarism},
            pdfauthor={Sridhar Adhikarla},
            pdfborder={0 0 0},
            breaklinks=true}
\urlstyle{same}  % don't use monospace font for urls
\usepackage{graphicx,grffile}
\makeatletter
\def\maxwidth{\ifdim\Gin@nat@width>\linewidth\linewidth\else\Gin@nat@width\fi}
\def\maxheight{\ifdim\Gin@nat@height>\textheight\textheight\else\Gin@nat@height\fi}
\makeatother
% Scale images if necessary, so that they will not overflow the page
% margins by default, and it is still possible to overwrite the defaults
% using explicit options in \includegraphics[width, height, ...]{}
\setkeys{Gin}{width=\maxwidth,height=\maxheight,keepaspectratio}
\IfFileExists{parskip.sty}{%
\usepackage{parskip}
}{% else
\setlength{\parindent}{0pt}
\setlength{\parskip}{6pt plus 2pt minus 1pt}
}
\setlength{\emergencystretch}{3em}  % prevent overfull lines
\providecommand{\tightlist}{%
  \setlength{\itemsep}{0pt}\setlength{\parskip}{0pt}}
\setcounter{secnumdepth}{0}
% Redefines (sub)paragraphs to behave more like sections
\ifx\paragraph\undefined\else
\let\oldparagraph\paragraph
\renewcommand{\paragraph}[1]{\oldparagraph{#1}\mbox{}}
\fi
\ifx\subparagraph\undefined\else
\let\oldsubparagraph\subparagraph
\renewcommand{\subparagraph}[1]{\oldsubparagraph{#1}\mbox{}}
\fi

%%% Use protect on footnotes to avoid problems with footnotes in titles
\let\rmarkdownfootnote\footnote%
\def\footnote{\protect\rmarkdownfootnote}

%%% Change title format to be more compact
\usepackage{titling}

% Create subtitle command for use in maketitle
\newcommand{\subtitle}[1]{
  \posttitle{
    \begin{center}\large#1\end{center}
    }
}

\setlength{\droptitle}{-2em}

  \title{Minimizing Programming Assignment Plagiarism}
    \pretitle{\vspace{\droptitle}\centering\huge}
  \posttitle{\par}
    \author{Sridhar Adhikarla}
    \preauthor{\centering\large\emph}
  \postauthor{\par}
      \predate{\centering\large\emph}
  \postdate{\par}
    \date{September 28, 2018}


\begin{document}
\maketitle

\subsubsection{Abstract}\label{abstract}

Technology helps students develop but it also attracts plagiarism.
Availability of electronic material through the internet has a great
positive impact on students, they get to learn a lot, and on the
downside also increases the risk of intentional or unintentional
plagiarism among students. Plagiarism is not just confined to essay or
text-based assignments. Programming assignments, being a part of most
computer-based courses, have a different kind of plagiarism and it is
difficult to detect. This paper explores the different reasons leading
to plagiarism in programming assignments and examines ways we can detect
it.

\subsubsection{Introduction}\label{introduction}

In today's world, we can get any information we want, be it medical
diagnosis or directions to go somewhere, over the internet. Same is the
case with programming assignments, students can get all the source code
for the assignment over the internet, creating a risk of intentional or
unintentional plagiarism. Internet can be a great source of knowledge
for the ones who look for it, but if someone just copies the source code
for the assignment over the internet, they are not learning anything
from it and are failing the sole purpose of the assigned task Plagiarism
in programming assignments is one of the most common things in computer
based courses, and it is increasing every year. Although it is not clear
if the increase is due to the improved algorithm for plagiarism checking
or more number of students involving in plagiarism. Students should know
the value and purpose of the task assigned to them, and they should also
be aware of how to prevent themselves from plagiarism. This is the
reason teachers should create awareness not just about plagiarism in
text-based assignments but also for programming assignments. Students
should be taught what is right and what is wrong.

Plagiarism in programming assignments is different from the text-based
assignments, and there is no common definition for source-code
plagiarism. According to Faidhi and Robinson, ``plagiarism occurs when
programming assignments are copied and transformed with very little
effort from the students'', whereas Joy and Luck define plagiarism as,
``unacknowledged copying of documents and programs''. A lot of
institutions use text similarity-based plagiarism detection tools for
programming assignments, which is not an effective way to do it. The
lexical elements of the code can be modified while retaining the
structure and logic of copied source code, and a text similarity
plagiarism detector cannot detect that. In this paper we will also talk
about some of the software's for source code plagia-rism detection.

\subsubsection{Students' knowledge about plagiarism in
programming}\label{students-knowledge-about-plagiarism-in-programming}

\begin{verbatim}
There has been a lot of research on how to improve plagiarism detectors in the past few years and very few answering the main problem causing it, "Does the student know what constitutes as pla-giarism?". It is very important for the students to have a clear idea of what is right and what is wrong. Most institutions focus on teaching students about plagiarism in text-based assignments and not about plagiarism in programming code. The definition of plagiarism is very different between the two.  If the institutions start focusing on teaching the students, the correct ways to avoid plagiarism in program-ming assignments this would reduce a lot of unintentional plagiarism.

In the paper, "Source code plagiarism- A student's perspective", the author conducted a survey for students from computer background all over UK and Europe, to find out how much students knew about Plagiarism in Programming. There were 15 questions in the survey and for every right answer the student got +1 points and for every wrong answer the student got -1. The total number of students who participated in the survey was 770, of which 615 were from UK and 87 from Europe and the remaining from rest of the world. The mean score obtained by students in Europe and UK was just 7.6 out of 15. This gives us a clear idea that students lack knowledge in the field, or there is some con-fusion in them about what is plagiarism. The frequency plot for the scores obtained by the students shows clearly that only a few have a clear idea about the subject of plagiarism. 
\end{verbatim}

The findings give a clear message that, along with detection and
enforcement of appropriate penalties, it is also necessary to look
beyond to understand clearly why plagiarism is so widespread. This study
gives us a clear idea that for students writing program code, the
training given is not effective in rais-ing awareness of what
constitutes plagiarism. Some areas of confusion in student's mind are
debata-ble, such as self-plagiarism and different definitions of
plagiarism between institutions.

\subsubsection{Plagiarism detection in programming
languages}\label{plagiarism-detection-in-programming-languages}

\begin{verbatim}
Automated methods for finding plagiarism in student's source code submissions have been in use for a very long time and there are many available search engines and services for it. The most common and effective technique for plagiarism detection in source code is tokenizing the code and then searching pair of submissions for long common substrings. Some of the widely used services, that use this technique for plagiarism detection in source code, are MOSS and JPlag. These are the services that are used by majority of the institutions for detecting source code plagiarism in student submissions. Although this detection is well established, future research for improving these are going on and a lot of extensions have been made for these methods to improve results. PlaGate is one such extension created. It is a novel tool that can be integrated to the existing detection tool to improve the performance. PlaGate implements a new approach for investigating the similarities between source code with a view to gathering evidence for proving plagiarism.

MOSS (Measure of Software Similarity), developed in 1994, is an automatic system for de-tecting similarity of programs. MOSS was created for detecting similarities between software's, but to date that main application of MOSS is detecting plagiarism in programming classes at institutions. The input to MOSS is a set of programs, and it returns references, the teacher must manually check through the references to confirm who is cheating. The only weakness of MOSS is that it does not do web search, but this is not a big problem as it common that for a large class more than one student copies code from the same source. MOSS can easily catch such similarities. On applying extensions like PlaGate to MOSS we will get exact results for students involved in cheating with evidence. 

[] JPlag is another such service for source code plagiarism detection. It is written in Java pro-gramming language and this works only with programming languages that are similar in syntax to Ja-va. JPlag is mostly used to plagiarism detection with languages like Java, C, C++. Input to JPlag is a set of programs. It compares these programs pairwise, computing total similarity value for each pair and a set of similarity regions. [] JPlag converts each program into a string of canonical tokens. [] For the comparison of two programs, JPlag then covers one such token string by substrings taken from the other where possible.
\end{verbatim}

\subsubsection{Conclusion}\label{conclusion}

\begin{verbatim}
Programming assignments are important part of a course curriculum as they give the practical experience to the students. It plays a key role in shaping the learning curve of students. I think if prop-er knowledge is not provided to the students about plagiarism in programming assignments and ap-propriate techniques are not used for detecting plagiarism in source code, the students are the one who suffer. They miss out on the most important part of the learning curve of the course. Students should be made aware of the consequences of plagiarism, that would motivate them to put some effort on their assignments.

If the students are made aware of the techniques to use to avoid plagiarism in programming and proper services, like MOSS and JPlag, are used to detect plagiarism in programming assignments, this would reduce plagiarism and even benefit the learning of students. 
\end{verbatim}

\subsubsection{References}\label{references}


\end{document}
